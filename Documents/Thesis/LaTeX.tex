%\documentclass[twoside]{pwrthesis}
\documentclass[twoside]{iisthesis}
% ---
\usepackage[MeX]{polski}
\usepackage[cp1250]{inputenc}
\usepackage{graphicx}

% Dodane przeze mnie d
\usepackage{fancyvrb} % dla srodowiska Verbatim
\usepackage{color}
\usepackage{lscape}


% definicje kolorow
\definecolor{ciemnoSzary}{rgb}{0.15,0.15,0.15}
\definecolor{szary}{rgb}{0.5,0.5,0.5}
\definecolor{jasnoSzary}{rgb}{0.2,0.2,0.2}

% Konfiguracja verbatima
\fvset{
	frame=single,
	numbers=left,
	fontsize=\footnotesize,
	numbersep=12pt,
%	framerule=.5mm,
	rulecolor=\color{ciemnoSzary},
%	fillcolor=\color{jasnoSzary},
	framesep=4pt,
	stepnumber=1,
	numberblanklines=false,
	tabsize=2,
%	formatcom=\color{szary}
}

\begin{document}

\title{Tytu� Pracy Magisterskiej}
\author{Imi� Nazwisko}
\advisor{dr hab. in�. Promotor, prof. PWr}
\instituteLogo{logos/pwr}
\slowaKluczowe{pierwsze\\drugie\\trzecie}

\date{\number\the\year}

% Wstawienie abstractu pracy
	%\input {abstract}
	
\abstractSH{
Bardzo kr�tkie streszczenie w kt�rym powinno si� znale�� om�wienie tematu pracy i poruszanych termin�w. Tekst ten nie mo�e by� zbyt d�ugi. }

\abstractPL{
	Abstrakt
}
\abstractEN{
	Abstrakt
}

\maketitle
\textpages

\chapter*{Wprowadzenie}
\chapter{Rozdzia� pierwszy}
\chapter{Rozdzia� drugi}

\appendix
\chapter{Co� dodatkowego}
\pagestyle{plain}
 
\listoffigures
\listoftables

\bibliographystyle{iisthesis}
\bibliography{mybibliography}

\end{document}

