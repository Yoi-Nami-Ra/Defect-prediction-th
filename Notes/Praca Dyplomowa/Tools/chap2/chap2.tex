% ********** Rozdzia� 2 **********
\chapter{CLUTO}
\label{sec:chapter2}

\section{Opis}

\section{U�ycie}

\textbf{vcluster} [optional parameters] \textit{MatrixFile NClusters}

Odpowiednio:
\begin{itemize}
	\item \textit{MatrixFile} - plik z macierz� danych.
	\item \textit{NClusters} - liczba klastr� w macierzy
	\item \textbf{parametry}
	\begin{itemize}
		\item \textbf{-clmethod=nazwa} - metoda klasteryzacji.
		\begin{itemize}
			\item \textbf{rb} - k-way - repeated bisections
			\item \textbf{rbr} - almost the same
			\item \textbf{direct} - wszystkie klastry s� wynajdywane r�wnocze�nie
			\item \textbf{agglo} - klastry obliczane s� u�ywaj�c \textit{agglomerative}. Proces zostaje przerwany, gdy znalezionych zostanie k klastr�w
			\item \textbf{graph} - Tworzony jest graf najbli�szych s�siad�w. Nast�pnie znajduje si� klastry metod� min-cut.
			\item \textbf{bagglo} - 

		\end{itemize}
		\item \textbf{-sim=nazwa} - Funkcja podobie�stwa
		\begin{itemize}
			\item \textbf{cos}
			\item \textbf{corr}
			\item \textbf{dist}
			\item \textbf{jacc}
		\end{itemize}
		\item \textbf{-crfun=nazwa} - kryterium klasteryzacji
		\begin{itemize}
			\item \textbf{i1}
			\item \textbf{i2}
			\item \textbf{e1}
			\item \textbf{g1}
			\item \textbf{g1p}
			\item \textbf{h1}
			\item \textbf{h2}
			\item \textbf{slink}
			\item \textbf{wslink}
			\item \textbf{clink}
			\item \textbf{wclink}
			\item \textbf{upgma}
		\end{itemize}
		\todo{doda� te pi�kne wzory matematyczne z manuala}
		\item \textbf{-agglocrfun=string}
		\item[] Kontroluje ''. Mo�na u�y� tych samych warto�ci jak dla opcji \textbf{-crfun}
		\item \textbf{-fulltree}
		\item \textbf{-rowmodel}
		
	\end{itemize}
\end{itemize}
% ********** Koniec rozdzia�u **********
